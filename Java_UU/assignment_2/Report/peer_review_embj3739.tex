\documentclass[hidelinks]{article}

\usepackage{fullpage} % Package to use full page
\usepackage{parskip} % Package to tweak paragraph skipping
\usepackage{tikz} % Package for drawing
\usepackage{amsmath}
\usepackage{hyperref}
\usepackage{enumitem}
\usepackage{float}
\usepackage{ upgreek }
\usepackage[explicit]{titlesec}
\usepackage{graphicx}
\usepackage {verbatim}
\newcommand{\RNum}[1]{\uppercase\expandafter{\romannumeral #1\relax}}
%\titleformat{\section}{\normalfont\Large\bfseries}{}{0em}{#1\ \thesection}

\title{Peer-Review 2 \\ Objektorienterad programmering med Java}
\author{Emil Björklund - embj3739 \\ emilbjorklund@live.com \\ \\ Källkod: Daren Domingo}


\begin{document}

\maketitle 
\newpage

\section*{Inledning}
\begin {itemize}
\item IDE: Eclipse IDE v4.5.0
\item Java verision: JDK 8 update 191
\item OS: MacOS Mojave
\end{itemize}



\section*{Review}
Programmet var lätt att köra och exekverade utan problem. Följande testfall utfärdades och klarades: \\
\begin{itemize}
\item Normalt testfall: Input gavs efter spelets regler, tills människa vann.
\item Onormalt testfall: Ett högt antal sticks att börja med, gick att mata in upp till maxtal för Integer. Input är även skyddat för negativ inmatning.
\item Onormalt testfall: Spelaren försöker bryta mot reglerna som att ta bort mer en hälften eller mindre än 1.
\end{itemize}

Vissa scanner inputs från konsollen är skyddade med try catch funktionalitet som i klassen Player till exempel. I klassen Nm finns det dock scanner inputs som ej är skyddade med try catch något som skulle kunna implementeras för att öka tillförlitligheten hos programmet. Lösningen är relativt lik min egna då jag också använder rand från standardbibloteket för att skapa splummässiga drag från datorn. I detta progam så skyddas rand med try catch precis som i min implementation.
Koden är mycket lättläst med få kommentarer. Kommentarer är dock inte nödvändiga på många ställen då koden är självbeskrivande. Metoderna är döpta efter dess funktionalitet som t.ex. getSticks() och removeSticks(). Klasserna computer och player har båda superklassen common för dess gemensama implementation vilket är ett snyggt upplägg.
\\ \\
Delay funktionen för kan tas bort då programmet upplevs långsamt, programmeraren har dock en poäng med att man vill fördröja datorns drag för att skapa en bättre kännsla men jag anser att det är onödigt. Rapporten skulle kunna beskriva lite mer om hur man kör programmet som en readme. Vilken .class fil innehåller main-konstruktorn t.ex. Rapporten kan även delas upp i underrubriker för lättare läsning.
\\ \\
I sora drag är koden väldigt välskriven och definitivt en insperationskälla för senare arbete. Korta IF-satser, självbeskrivande variabler samt metoder och lagomt luftig kod anser jag är dom största styrkorna. Upplägget med klasser är mycket likt min egna implementation med en superklass, en klass för människa och dator samt en klass för själva spelet.  


\end{document}