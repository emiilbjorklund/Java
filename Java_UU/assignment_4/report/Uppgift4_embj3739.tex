\documentclass{article}
\usepackage{fullpage} % Package to use full page
\usepackage{parskip} % Package to tweak paragraph skipping
\usepackage{tikz} % Package for drawing
\usepackage{amsmath}
\usepackage{hyperref}
\usepackage{enumitem}
\usepackage{float}
\usepackage{ upgreek }
\usepackage[explicit]{titlesec}
\usepackage{graphicx}
\usepackage {verbatim}
\newcommand{\RNum}[1]{\uppercase\expandafter{\romannumeral #1\relax}}
%\titleformat{\section}{\normalfont\Large\bfseries}{}{0em}{#1\ \thesection}

\title{Inlämning 4 \\ Objektorienterad programmering med Java}
\author{Emil Björklund - embj3739 \\ emilbjorklund@live.com}


\begin{document}

\maketitle 
\newpage

\section{Inledning}
\begin {itemize}
\item IDE: Visual Studio Code
\item Java verision: JDK 8 update 191
\item OS: MacOS Mojave
\end{itemize}

\textbf{Note.} Källkoden bifogas i separat .zip som finns i inlämningen.

\section{Struktur}
Koden bygger på ett antal klasser beskrivna nedan. Strukturen bygger på given kod med mindre modifikationer för att passa tänkt lösning.
\subsection{Class Pasture}
Klassen Pasture har main konstruktorn 
\subsection{Class PastureGUI}
\subsection{Class Engine}
\subsection{Class Entity}
\subsubsection{Class Animal}
\paragraph{Class Carnivore}
\paragraph{Class Herbivore}
\subsubsection{Class Plant}
\subsubsection{Class Fence}
\subsection{Enum IsCompatible}
\subsection{Enum Constanta}
\end{document}