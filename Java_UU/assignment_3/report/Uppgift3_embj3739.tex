\documentclass[hidelinks]{article}
\usepackage{fullpage} % Package to use full page
\usepackage{parskip} % Package to tweak paragraph skipping
\usepackage{tikz} % Package for drawing
\usepackage{amsmath}
\usepackage{hyperref}
\usepackage{enumitem}
\usepackage{float}
\usepackage{ upgreek }
\usepackage[explicit]{titlesec}
\usepackage{graphicx}
\usepackage {verbatim}
\newcommand{\RNum}[1]{\uppercase\expandafter{\romannumeral #1\relax}}
%\titleformat{\section}{\normalfont\Large\bfseries}{}{0em}{#1\ \thesection}

\title{Inlämning 3 \\ Objektorienterad programmering med Java}
\author{Emil Björklund - embj3739 \\ emilbjorklund@live.com}


\begin{document}

\maketitle 
\newpage

\section*{Inledning}
\begin {itemize}
\item IDE: Visual Studio Code
\item Java verision: JDK 8 update 191
\item OS: MacOS Mojave
\end{itemize}

\textbf{Note.} Källkoden bifogas i separat .zip som finns i inlämningen.

\section*{Struktur}
\subsection*{Class Window}
Denna klass innehåller main konstruktor och anropas därmed av användaren vid start.
Klassen öppnar en "JFileChooser" för att användaren enkelt skall kunna importera önskade bilder.
Om användaren inte importerar ett jämnt antal bilder får användaren på nytt importera korrekt antal. Ett jämnt antal bilder
krävas för att programmet skall kunna applicera en framsida samt baksida av varje kort. 
När användaren har lyckats med sin importering av bilder sparas alla i en array av "Files". Denna array skickas dean vidare när main
deklarerar ett nytt "Window" objekt. \\
I objektet "Window" skapas en "Jframe" samt ett objekt kallat Panel där all hantering av bilder och position sker.
Objektet panel läggs sedan till i "Jframe".

\subsection*{Class Panel}
I klassen Panel deklareras det två arrayer. Den första arrayen är en tom "Image" array med lika många platser som "Files" arrayen från main är.
Den andra arrayen är av typen boolean som innehåller information om respektive bild visar baksida eller ej.
När arrayerna är deklarerade anropas metoden "initLayers" som ser ut som följande:
\begin{verbatim}
private  void initLayers(File files[]) {
    for (int i = 0; i <= (files.length - 1)/2; i++) {
        layers[i] = new Image(i*10,i*20,files[i],files[i+(files.length)/2]);
    }
}    
\end{verbatim}
Denna loop itererar längden av "Files" arrayen. Varje possition i arrayen fylls med ett Objekt av klassen Image. 
Denna klass tar även hand om events som "MouseClicked", "MouseDraged", "MousePressed" för att kunna interagera med bilderna i programmet.
Logiken bakom "MouseClicked" och "MouseDraged" är nästan lika och ser ut som följande:
\begin{verbatim}
public void mousePressed(MouseEvent e) {
    for (int i = 0; i <= layers.length - 1; i++) {
        if (layers[i].constraint(e.getX(), e.getY())){
            Image temp = layers[i];
            
            for (int j = i; j > 0; j--){
                layers[j] = layers[j-1];
                isBackside[j] = isBackside[j-1];
            }
            
            layers[0] = temp;
            isActive = true;
            break;
        } 
    }
}
\end{verbatim}
När musen trycks ner eller klickas loopar en for-loop från noll till längden av bild arrayen.
Om något lagers bild omsluter kordinaterna för musen flyttas indexeringen om i arrayen så denna bild hamnar längst fram.
För att kolla om musen är över en bild anropas metoden "constraint" i klassen "Image"
När lagerna har bytt plats så att den aktiva bilden är längst fram i arrayen blir variablen "isActive" sann och 
möjliggör för eventet "MouseDraged" att anropa klassen "Images" metod action. Efter anropet kallas funktionen "repaint".
I klassen "Panel" finns metoden "paintCompoent" denna metod itererar arrayen för bilder baklänges så att den aktiva bilden målas upp sist.

\subsection*{Class Image}
Klassen Image har all information om bilden och dess possition. När klassen deklareras som ett objekt från klassen Panel anropar konstrukorn metoden
"readImage" denna metod läser in två bilder från sökvägen som ges som array i metod-huvudet. Koden kan ses nedan.
\begin{verbatim}
public  void readImage(File file1,File file2) {
    try
    { 
        image = new BufferedImage(100, 100,BufferedImage.TYPE_INT_ARGB);
        image_backside = new BufferedImage(100, 100,BufferedImage.TYPE_INT_ARGB);
        image = ImageIO.read(file1); 
        image_backside = ImageIO.read(file2); 

        System.out.println("Reading complete."); 
    } 
    catch(IOException e) 
    { 
        System.out.println("Error: "+e); 
    } 
}
\end{verbatim}
Klassen har även variabler som har information om bildens possition uttryckt i x och y kordinater.
Om ett mus event händer i klassen Panel anropas klassen Images metod constraint för att verifiera om musens kordinater är inom bildens. Vidare har klassen även
en metod där bildens kordinater uppdateras beroende på musens kordinater för att sedan kunna ritas om.



\end{document}